\thispagestyle{empty}
\setstretch{1.15} % Zeilenspacing
\section*{Zusammenfassung}

\bigskip 


In der vorliegenden Bachelorarbeit werden Algorithmen basierend auf Monte-Carlo-Tree-Search und AlphaZero verglichen. Die Algorithmen werden auf das Brettspiel Reversi angewendet. Das Ziel dieser Arbeit ist die Überprüfung einer Verbesserung dieser Kombinationen, bestehend aus konventioneller Implementierung und maschinellen Lernen, gegenüber dem klassischen Monte-Carlo-Tree-Search. Es werden vier Hybrid-Algorithmen, bestehend aus Kombinationen von Monte-Carlo-Tree-Search und MiniMax, ausgewertet und ihre Leistung anhand ausgewählter Parameter verglichen.\\
Anschließend wird AlphaZero, eine Kombination aus einem neuronalen Netz und Monte-Carlo-Tree-Search, vorgestellt, ebenfalls auf Reversi angewendet und dessen Ergebnisse ausgewertet.\\
Die Umsetzung und Implementierung dieser Algorithmen wird hinzukommend beschrieben.\\
Es konnte gezeigt werden, dass die Kombinationen mit MiniMax an verschiedenen Stellen des Algorithmus nicht zu auffallenden Verbesserungen beigetragen haben. Die Siegesraten waren nicht besser wie die des Basisalgorithmus Monte-Carlo-Tree-Search.\\
Die Umsetzung des Lernens des Spiels Reversi mit einem residualen Netz konnte umgesetzt werden. Dessen Auswertung führte bei einer trivialen KI als Gegner zu einer Siegesraten von über 70\%. Das Ergebnis wird abschließend anhand eines Vergleichs mit aus der Literatur bekannten neuronalen Netzen, welche Faltungsnetze verwenden, verglichen und eingeordnet.

\section*{Abstract}

\bigskip 


In this bachelor thesis, algorithms based on Monte-Carlo-Tree-Search and AlphaZero are compared. The algorithms are applied to the board game Reversi. The goal of this work is to verify an improvement of these combinations, consisting of conventional implementation and machine learning, over classical Monte-Carlo Tree Search. Four hybrid algorithms, consisting of combinations of Monte-Carlo Tree Search and MiniMax, are evaluated and their performance is compared based on selected parameters.\\
AlphaZero, a combination of a neural network and Monte-Carlo Tree Search, is then presented, also applied to Reversi, and its results evaluated.\\
The realization and implementation of these algorithms is described in addition.\\
It was shown that the combinations with MiniMax at different points of the algorithm did not contribute to striking improvements. The win rates were not better than those of the basic Monte Carlo Tree Search algorithm.\\
The implementation of learning the game Reversi with a residual network could be implemented. Its evaluation led to a win rate of over 70\% with a trivial AI as opponent. Finally, the result is compared and classified with neural networks known from the literature, which use convolutional networks.